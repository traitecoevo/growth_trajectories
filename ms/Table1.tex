\documentclass[9pt,onecolumn]{pnas-new}

\newcommand{\plant}{\texttt{plant}}
\newcommand{\wplcp}{\textsc{wplcp}}
\newcommand{\lma}{\textsc{lma}}
\newcommand{\wood}{\textsc{wd}}
\newcommand{\seed}{\textsc{sm}}
\newcommand{\hmat}{\textsc{h}$_{\rm mat}$}
\newcommand{\nitrogen}{\textsc{n}}

  
\begin{document}

\begin{table}[!ht]
\caption{Empirical phenomena explained in this paper.}
\begin{tabular}{p{8cm}}
\toprule
\textbf{Change in growth rate with increasing size (Fig. 1)}\\
 $\,$ Biomass growth: Hump-shaped (22, 23) \\
 $\,$ Plant mass: Increasing (24, 25) \\
 $\,$ Height: Hump-shaped (24, 26, 27) \\
 $\,$ Stem-diameter: Hump-shaped (10, 28, 29) \\
 $\,$ Relative growth rate (all variables): decreasing (19, 30)\\
\textbf{Effect of traits on growth rate (Fig. 3)}\\
 $\,$ {\seed}: $\downarrow$ values $\downarrow$ seedling size, \& thus $\downarrow$ absolute \& $\uparrow$ relative growth rate (21) \\
 $\,$ $H_{\rm mat}$: $\downarrow$ values $\downarrow$ growth rate at at larger sizes (21)\\
 $\,$  {\nitrogen}: $\downarrow$ values $\uparrow$ growth rate irrespective of size, but only in high light (21)\\
 $\,$ {\lma}: $\downarrow$ values $\uparrow$ growth rate when small, not at mid-large sizes (21)\\
 $\,$ {\wood}: $\downarrow$ values $\uparrow$ growth rate, except at largest sizes (21)\\
\textbf{Responsiveness of growth rate to changes in light, $E$ (Fig. 3)}\\
 $\,$ {\nitrogen}: $\uparrow$ values respond more \\
 $\,$ {\lma}: $\downarrow$ values respond more \\
 $\,$ {\wood}: $\downarrow$ values respond more (18)\\
\textbf{Shade tolerance, \textsc{wplcp} (Fig. 4)}\\
 $\,$ Size: decreasing  (22, 31, 32)\\
 $\,$ {\nitrogen}: $\downarrow$ values $\uparrow$ shade tolerance (33–35)\\
 $\,$ {\lma}: $\downarrow$ values $\downarrow$ shade tolerance (9, 32, 33, 35)$^3$ \& $\uparrow$ \textsc{lai} (15, 36, 37) \\
 $\,$ {\wood}: $\downarrow$ values $\downarrow$ shade tolerance (38)\\
  \bottomrule
\end{tabular}

\addtabletext{{\seed} = seed mass, {\hmat}= height at maturation, {\nitrogen}= leaf nitrogen content per unit leaf area, {\lma}=leaf mass per unit leaf area, {\wood}=wood density, \textsc{lai} = leaf area index. Footnotes: $^1$ Similar responses are predicted for maximum photosynthetic rate per leaf area \& dark respiration rate per leaf area; here both related to {\nitrogen}. $^2$ Similar responses are predicted for leaf lifespan; here directly related to {\lma}. $^3$ (35) finds a relationship between \textsc{wplcp} \& leaf respiration rate expressed per unit leaf area or per unit leaf mass.

}
\label{tab:phenomena}
\end{table}

\end{document}
