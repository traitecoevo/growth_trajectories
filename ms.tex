\documentclass[12pt, a4paper]{article}
\RequirePackage[hmargin=2cm,vmargin=2cm]{geometry}

\usepackage[T1]{fontenc}
\usepackage{lmodern}
\usepackage{amssymb,amsmath}
\usepackage{authblk} % allows for better author affiliations

% Fancy HEADER
\usepackage{fancyhdr}
\pagestyle{fancy}
\pagenumbering{arabic}
\lhead{\itshape{\nouppercase{\leftmark}}}
\chead{}
\rhead{\thepage}
%\lfoot{v }
\cfoot{}
%\rfoot{\thepage}

\usepackage{natbib}
\bibliographystyle{ecol_let}
\setcitestyle{authoryear,open={(},close={)}}

\usepackage{graphicx}
% We will generate all images so they have a width \maxwidth. This means
% that they will get their normal width if they fit onto the page, but
% are scaled down if they would overflow the margins.
\makeatletter
\def\maxwidth{\ifdim\Gin@nat@width>\linewidth\linewidth
\else\Gin@nat@width\fi}
\makeatother

\let\Oldincludegraphics\includegraphics
\renewcommand{\includegraphics}[1]{\Oldincludegraphics[width=\maxwidth]{#1}}

\usepackage[rgb,dvipsnames]{xcolor}
\definecolor{grey}{rgb}{0.5, 0.5, 0.5}

\usepackage[setpagesize=false, % page size defined by xetex
              unicode=false, % unicode breaks when used with xetex
              xetex]{hyperref}
\hypersetup{breaklinks=true,
            bookmarks=true,
            pdfauthor={},
            pdftitle={Untangling the link between traits, size and growth rate in plants},
            colorlinks=true,
            citecolor=black,
            urlcolor=grey,
            linkcolor=grey}
\setlength{\parindent}{0pt}
\setlength{\parskip}{6pt plus 2pt minus 1pt}
\setlength{\emergencystretch}{3em}  % prevent overfull lines

\title{\LARGE Untangling the link between traits, size and growth rate in plants}

\author[1]{Daniel S. Falster}
\author[1]{Richard G. FitzJohn}
\author[2]{Joe Wright}

\affil[1]{{\footnotesize Biological Sciences, Macquarie University, North Ryde, NSW 2109, Australia}}
\affil[2]{{\footnotesize Center for Tropical Forest Science, Smithsonian Tropical Research Institute, Panama, Republic of Panama}}
\renewcommand\Authands{ and }
\date{\vspace{-3em}}

\begin{document}

\maketitle
\thispagestyle{empty} % avoid header and footer on first page. Put after \maketitile, if using that

\section*{Abstract}\label{abstract}

\section*{Introduction}\label{introduction}

Functional traits are thought to capture core differences in the strategies
plants use to generate and invest surplus energy
\citep{wright_world-wide_2004,chave_towards_2009,westoby_plant_2002}. Useful
traits are those that
can be easily measured for diverse species and where differences between
species are large compared to intra-specific variation. Moroever, the uptake
of traits reflects the observation that although most plants have same basic
construction, physiological function and resource requirements, species differ
considerably from one another in rates of carbon, nitrogen and water uptake,
growth, mortality and seed production. Data for prominent traits now exists
for many of the world's 250000 plant species \citep{cornwell_functional_2014}.
The effect of some traits of multiple aspects of plant function has also been
well documented, including mechanical strength, biomass partitioning, rates of
photosynthesis and rates of tissue turnover
\citep {wright_world-wide_2004,chave_towards_2009}. As a result, the idea has
taken hold that the growth startegy and ecology of a plant species might usefully
be characterised simply by knowing its traits.

While the influence of plant traits on elements of plant physiologival
function is increasingly understood, attempts at using traits to predict
demographic rates have only met with mixed success. In seedlings, the trait
leaf mass per area (LMA) -- a central part of the leaf economics spectrum --
is tightly related to plant relative growth rate in mass. LMA and its
correlate leaf lifespan have also been linked to height growth rate for small
seedlings and saplings. These early successes prompted researchers to search
for similar relationships in large plants. However, the results showed that in
large plants, LMA was not correlated with diameter growth rate. Meanwhile,
other traits such as wood density showed strong relationships
to growth in large plants, but only weak relationships in seedlings \citep
{castro-diez_stem_1998}. To date each relationship has been quantified mostly for
one or few traits and at a single life-stage. Yet the picture is emerging is that the
link between traits and growth may be modified by plant size; a finding which would
challenge the asummption applied by many that particular ends of the trait spectrum
translate directly into fast or slower growth.

The challenge of interpreting diverse empirical results linking traits to
growth rate -- or not -- would clearly be easier if we had clearl expectations
on what signal one should expect, given our understanding of how traits
infleunce plant function. This is a primary, yet perhaps underappreciated,
role for theory: to lay expectations based on mechistic principles against
which empircal data can be compared. Current empircial results suggest the
effect of traits on growth changes with size and possibly also light
environment, yet current theory says little about how relationships might come
about. A widely-used model suggests mass-based relative growth rate will be a
linear function of LMA, at least for seeldings.  An extension of model
suggests a similar relationship may hold at larger sizes. But this prediction
doesn't empirical results. Meanwhile, theoretical predictions on how other
traits should influence growth are largely absent.

A further problem with existing models is that the effects of traits are all
realised via mass production, whereas the effect of some prominent traits is
most likely on the quality of tissue construction, or the amoutn of mass
allocated to different tissues. There are two problem here. First, measuring mass
production is only practical for small plants that can be harvested. For larger plants,
growth is more often emasured either as diameter or height growth. For example, the
CTFS monitors diameter growth in 49 large permenenat plots worldwide. Establishing
the link between traits, mass growth and diamater growth is thus essential for predicting
the effect of traits on growth across these diverse datasets.

Here we provide a mechanistic framework describing how traits link to
demography, linking to commonly used metrics such as height and diameter
growth, combine this with fresh empirical insight using largest
collation of growth and trait data available, from long-term plot at
BCI, Panama.

\begin{itemize}
\itemsep1pt\parskip0pt\parsep0pt
\item
  Shows why effect of traits changes with size,size dependent effect
  differs among traits.

  \begin{itemize}
  \itemsep1pt\parskip0pt\parsep0pt
  \item
    Contradicts widespread assumption that traits relate to growth in
    simple linear way
  \end{itemize}
\item
  Offers framework for interpreting and understanding empirical results
\end{itemize}

\section*{Fresh empirical motivation}\label{fresh-empirical-motivation}

To help motivate our model, we begin with a fresh analysis of growth dynamics
in the long-term plot at Barro Colorado Island in Panam, where the growth
rates for over 100 species have been recorrded over the last 30 years. A
prior analysis by \citet{wright_functional_2010} found a mdoerate effect of
wood density on growth rate, but almsot no effect of LMA, and only a weak
effect of a thrid trait maximum plant height (HMAX).


Fresh empirical analysis highlights how effect of traits on pot growth
changes with size.

\begin{itemize}
\itemsep1pt\parskip0pt\parsep0pt
\item
  Effect of lma stronger than previously thought,
\item
  Bridges from seedling studies to adults
\end{itemize}

Results stronger than previously reported.

\section*{Materials and Methods}\label{materials-and-methods}

\subsection*{Conceptual framework}\label{conceptual-framework}

Builds on 2011 paper - highlight what's new from there.

Decomposition of diameter and height growth

Functional-balance Model of growth

\begin{itemize}
\itemsep1pt\parskip0pt\parsep0pt
\item
  four key assumptions.
\item
  simple linear functions non-critical for results of this paper,
\item
  explore using an example model, but any model with basic size effects
  will display same behaviour

  \begin{itemize}
  \itemsep1pt\parskip0pt\parsep0pt
  \item
    details about our model ()
  \item
    show argument hold intuiatively
  \end{itemize}
\end{itemize}

Embedding trait-related trade-offs

\begin{itemize}
\itemsep1pt\parskip0pt\parsep0pt
\item
  use table with arrows to
\end{itemize}

Connection to earlier work

\begin{itemize}
\itemsep1pt\parskip0pt\parsep0pt
\item
  generalised version
\item
  present old model as dp/dt so consistent with ours
\end{itemize}

\subsection*{Analysis}\label{analysis}

\subsection*{Growth data}\label{growth-data}

\section*{Results}\label{results}

\subsection*{Changes in growth rate with
size}\label{changes-in-growth-rate-with-size}

\subsection*{How traits influence growth
rate}\label{how-traits-influence-growth-rate}

\subsubsection*{LMA}\label{lma}

\subsubsection*{Wood density}\label{wood-density}

\subsubsection*{HMax}\label{hmax}

\section*{Discussion}\label{discussion}

\begin{itemize}
\itemsep1pt\parskip0pt\parsep0pt
\item
  mechanistic model of height and dbh growth which accounts for effects
  of size, light environment and traits

  \begin{itemize}
  \itemsep1pt\parskip0pt\parsep0pt
  \item
    need to account for size long recognized, but not straightforward
  \item
    look at growth in wide range of measures: mass, height, dbh
  \end{itemize}
\item
  Compared to previous work, shows effects of traits more about
  allocation than mass production
\end{itemize}

\subsection*{Tree growth is more than just
photosynthesis}\label{tree-growth-is-more-than-just-photosynthesis}

Recently realisation that allocation and turnover major areas of
uncertainty

Model reconciles idea that traits can have negative impact on carbon
budget, but still have positive influence on growth rates

\begin{itemize}
\itemsep1pt\parskip0pt\parsep0pt
\item
  Enquist 2007: sla and wd increase mass-production, where as in our
  theory effects on mass production are negative. (also contradicts
  earlier model from Enquist 1999, where argued wood density had no
  effect on mass production)
\end{itemize}

Assumes low lma and wd used to reduce costs of building leaf and stem,
ie. total leaf area and stem corss section remains same. In this way
differs to assumptions of recent work by anten\_role\_2010,
larjavaara\_rethinking\_2010.

Model is realisation of several ideas that have been around for some
time.

\subsection*{Towards a global model of tree
growth}\label{towards-a-global-model-of-tree-growth}

Key patterns explained

Data challenges

Reproductive allocation

\subsection*{Detecting trait signals in
data}\label{detecting-trait-signals-in-data}

Need more than simply look for correlation with mean growth rate

\begin{itemize}
\itemsep1pt\parskip0pt\parsep0pt
\item
  understand influences of size on trait of interest
\item
  traits define potential growth rate --\textgreater{} how to extract
  this
\item
  ideally start to use mechanistic models
\end{itemize}

\subsection*{Implications for trait-based
approaches}\label{implications-for-trait-based-approaches}

How do comparative people interpret sla now - beyond fast slow.

Think about traits as defining potential trajectory, rather than making
species fast or slow growing per se.

Selection on traits in different parts of lifecycle

Trait plasticity - can explain, no clear advantage of low lma at large
sizes

Framework for generalising about effect of traits

\section*{Conclusion}\label{conclusion}

\newpage

\section*{Tables}\label{tables}

\textbf{Table 1 - Key equations of growth model}

\begin{itemize}
\itemsep1pt\parskip0pt\parsep0pt
\item
  key parameters of model
\end{itemize}

\textbf{Table 2 - effect of traits on demography}

\begin{itemize}
\itemsep1pt\parskip0pt\parsep0pt
\item
  combines previous tables 1 \& 2
\end{itemize}


\newpage

\section*{Figures}\label{figures}

\begin{figure}[htbp]
\centering
\includegraphics{figures/BCI_data2.pdf}
\caption{\textbf{The relationship between traits and potential growth
rate varies with plant size.} For XXX species from tropical rain forest
in Panama, we estimated the potential growth rate of individual's in
each species at a series of diameters \(D\), indicated along right hand
side. The size of circles in each panel indicates the number of data
points used to estimate potential growth rate. Traits values were
calculated from representative individuals in each species.
\label{f-BCI}}
\end{figure}

\newpage

\begin{figure}[htbp]
\centering
\includegraphics{figures/SI_size_dhdt.pdf}
\caption{\textbf{Conceptual framework for understanding size-dependent
changes in growth rate.} \label{f-conceptual}}
\end{figure}

\newpage

\begin{figure}[htbp]
\centering
\includegraphics{figs/allometry.pdf}
\caption{\textbf{Key assumptions of a functional balance and trait
trade-offs model, evaluated using global dataset.} We used the biomass and
allometry database to evaluate model assumptions about \textbf{a,}
scaling of leaf area with plant height, \textbf{b} Scaling of sapwood
area with leaf area, and \textbf{c} scaling of root mass with leaf area.
Each dot is a single plant. Lines show standardised major axis lines
fitted to data from each site, with intensity of shading adjusted
according to strength of the relationship. Colours indicate vegetation
type. Dashed black lines show values expected under functional-balance
assumption (see Supplementary text for details). \label{f-assumptions}}
\end{figure}

\newpage

\begin{figure}[htbp]
\centering
\includegraphics{figures/growth_light.pdf}
\caption{\textbf{Traits moderate the responsiveness of growth to changes
in light environment.} Panels show predicted relationship between
specific trait and diameter growth rate, for a plant of specified
diameter and under a range of shading environments.
\label{f-growth_light}}
\end{figure}

\newpage

\begin{figure}[htbp]
\centering
\includegraphics{figures/max_leaf_above.pdf}
\caption{\textbf{The effect of traits on growth changes with size and
light environment.} Traits moderate the responsiveness of growth to changes in light
environmentThis figure needs to show more of key result with respect to to size.
Possibly separate figure. \label{f-shifts}}
\end{figure}

\newpage

\begin{figure}[htbp]
\centering
\includegraphics{figures/max_leaf_above.pdf}
\caption{\textbf{Low construction cost leads to shade intolerance,
because of costs of high turnover.} Panels show effect of traits on
maximum amount of shading that can be endured before net production (eq.
\ref{eq:dPdt}) reaches zero. Lines indicate relationship for plants of a
given height. \label{f-wplcp}}
\end{figure}

\newpage

\bibliography{references}
\end{document}